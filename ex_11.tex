\section{Exercise 11}

A theory $T$ is called maximally consistent if all theories $T' \supset T$ are inconsistent.

\begin{enumerate}
	\item Prove that a theory is maximally consistent iff it is complete and consistent.\\
	
	\noindent
	\textbf{Solution.} 
	Per the definition of consistency for a theory $T$, $T$ is consistent iff there is no sentenced $F$ s.t. both $T \vdash F$ and $T \vdash \neg F$.
	It is complete iff, for every sentence $F$, $T \vdash F$ or $T \vdash \neg F$ hold.\\
	
	\textbf{The $\Rightarrow$-direction.} Let $T$ be a complete and consistent theory and let $T'$ be its superset. $T'$ being a superset implies that $F \in (T' - T)$ for some sentence $F$. Since $F$ is not in $T$ and $T$ is complete, $T \vdash \neg F$ must hold. That, in turn, implies that $\neg F \in T'$, making $T$ inconsistent.\\
	
	\textbf{The $\Leftarrow$-direction.} Let $T$ be a maximally consistent theory. We show its completeness:
		\begin{itemize}
			\item[Completeness] Suppose there is a sentence $F$ s.t. $F \notin T$ and $\neg F \notin T$. Then we could add either to $T$, obtaining a consistent superset. This contradicts the definition of maximal consistency.
		\end{itemize}
		
	Consistency, in general, doesn't hold for maximally consistent theories as defined here. Suppose that $T = \{ F | F \mt{ is a sentence over } \mathcal{L} \}$. Then no theory $T'$ which is a strict superset of $T$ exists and therefore, ``all`` superset-theories of $T$ are inconsistent, but $T$ itself is not consistent. Consistency therefore has to be added to the definition of ``maximally consistent''.
%\end{itemize}

	\item Assume that $T$ is a theory over a language $\mathcal{L}$ containing =. Prove that the following are equivalent: (a) $T$ is inconsistent, (b) $T = \{F | F \mt{ sentence over } \mathcal{L} \}$, (c) $(\exists x x \neq x) \in T$.\\
	
	\textbf{Solution.} I shall prove this via a cycle of implications.
	
	\begin{enumerate}
		\item[(a) $\Rightarrow$ (b)]. If $T$ is inconsistent, there exists a sentence $G$ s.t. $G \in T$ and $\neg G \in T$. From the assumptions $G,\neg G$, any sentence $F$ follows, as this LK-proof shows:
		
		\begin{prooftree}
		\AxiomC{$G \vdash G, F$}
		\RightLabel{$\neg$-l}
		\UnaryInfC{$G, \neg G \vdash F$}
		\end{prooftree}
		
		Since theories are, by definition, deductively closed, $F$ must also be in $T$. That, in turn, is exactly the definition (b).
		
		\item[(b) $\Rightarrow$ (c)] The implication trivially holds, since $(\exists x. x \neq x)$ is a sentence of any language that includes =.
		
		\item[(c) $\Rightarrow$ (a)]. $\exists x. x \neq x$ is defined as $\exists x. \neg (x = x)$, which can be further reduced to $\neg (\forall x. x = x)$. This is a direct negation of the reflexivity axiom: $\forall x. x = x)$. Thereby, $T$ is inconsistent.
	\end{enumerate}
\end{enumerate}

